% !TeX encoding=utf8
% !TeX spellcheck = de_CH_frami

%%% --- Acronym definitions
\IfDefined{newacronym}{%
	\newacronym{zhaw}{ZHAW}{Zürcher Hochschule für Angewandte Wissenschaften}
	\newacronym{crm}{CRM}{Customer Relationship Management}
	\newacronym{weka}{Weka}{Waikato Environment for Knowledge Analysis}
	\newacronym{pam}{PAM}{Partitioning Around Medoids}
	\newacronym{dbscan}{DBSCAN}{Density-Based Spatial Clustering of Applications with Noise}
}

% use it with \gls{glos:DVD}
% use plural with \glspl{thinClientLabel}

%%% --- Symbol list entries

%\newglossaryentry{symb:Pi}{%
%  name=$\pi$,%
%  description={mathematical constant},%
%  sort=symbolpi, type=symbolslist%
%}


%%% --- Glossary entries
\newglossaryentry{irent}{name=Irent,
	description={Das Irent ist eine Applikation innerhalb der Hotelplan Management AG, welche für das Customer Relationship Management verwendet wird. Es ist eine Erweiterung für SAP und speichert Kunden- sowie Buchungsdaten für die Interhome AG.}
}
\newglossaryentry{supervisedlearning}{name=überwachtes lernen,
	description={Überwachtes lernen ist ein Teilgebiet vom maschinellen Lernen. Beim lernen werden zusammenhänge zwischen Eingaben und Ausgaben aufgedeckt. Es handelt sich dabei um einen überwachten Prozess, wenn die Ausgabewerte von vornherein in der Datenbasis bekannt sind.}
}
\newglossaryentry{unsupervisedlearning}{name=unüberwachtes lernen,
	description={Unüberwachtes lernen ist ein Teilgebiet vom maschinellen Lernen. Beim lernen werden zusammenhänge zwischen Eingaben und Ausgaben aufgedeckt. Es handelt sich dabei um einen unüberwachten Prozess, wenn die Ausgabewerte zu Beginn nicht bekannt sind und diese zuerst gebildet werden müssen.}
}
\newglossaryentry{gls:gnugpl}{name=GNU General Public License,
	description={GNU GPL ist eine Software Lizenz, welche es erlaubt, den Code auszuführen, zu studieren, zu teilen sowie zu modifizieren. Weitere Informationen unter \url{gnu.org/licenses/gpl.html}}
}
\newglossaryentry{acr:gnugpl}{
	type=\acronymtype, 
	name={GNU GPL}, 
	description={GNU General Public License}, 
	first={Application GNU General Public License (GNU GPL)\glsadd{gls:gnugpl}}, 
	see=[Glossar:]{gls:gnugpl}}



%\newglossaryentry{glos:failureSafetyLabel}{name=Ausfallsicherheit,
%	description={Mit der Ausfallsicherheit wird die minimale zeitliche Erreichbarkeit (resp. maximale Ausfallzeit) eines Systems angegeben. Ist diese Ausfallzeit sehr gering spricht man von Hochverfügbarkeit (High Availability), dazu ist mindestens eine Verfügbarkeit von 99.9 \% nötig.
%	Die Verfügbarkeit berechnet man wie folgt:
%	\begin{gather*}
%		\text{Verfügbarkeit} = (1- \frac{\text{Ausfallzeit}}{\text{Periode}}) * 100\\
%		\text{Ausfallzeit} = (1 - \frac{\text{Verfügbarkeit}}{100}) * \text{Periode}
%	\end{gather*}}
%}
%\newglossaryentry{glos:thinClientLabel}{
%  name=Thin Client,
%  plural={Thin Clients},
%  description={Ein Thin Client ist ein günstiger, rechen-schwacher Computer. Er wird dazu verwendet, um Arbeiten zu erledigen die auf einem rechen-starken Server statt finden. Ein Thin Client übernimmt hauptsächlich die Bereitstellung von }
%}




