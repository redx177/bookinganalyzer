
\chapter{Konzept}
\label{sec:konzept}

\section{Vorgehensweise}
\label{sec:konzept:vorgehensweise}
Die Umsetzung in diesem Projekt wird zweistufig durchgeführt. Als erstes gibt der User eine Abfrage ein, für welche anschliessend eine eine Liste von häufig auftretenden Attributen gesucht werden soll (siehe \cref{sec:einletung:ziel} \nameref{sec:einletung:ziel}). Die Abfrage schränkt dabei den Datenbestand ein. Dafür kann der erste Schritt des \nameref{sec:recherche:dataminingtechniken:disziplinen:association} eingesetzt werden. Für die Analyse der Restmenge eignet sich die \nameref{sec:recherche:dataminingtechniken:disziplinen:clusteranalysis}, da zu Beginn die Klassen noch nicht bekannt sind. Dadurch fällt \nameref{sec:recherche:dataminingtechniken:disziplinen:classification} weg. Die \nameref{sec:recherche:dataminingtechniken:disziplinen:regression} eignet sich nicht, da dadurch nur numerische Werte vorausgesagt werden können, in dieser Arbeit jedoch Klassen vergeben werden müssen. \nameref{sec:recherche:dataminingtechniken:disziplinen:collaborativefiltering} versucht durch Kundenbewertungen ähnliche Objekte vorzuschlagen. Dies kann für einen Recommender eingesetzt werden, jedoch nicht für die Auffindung von häufigen Attributen.

% Assoziationsanalyse
%Als erstes werden Attributkombinationen (Mengen von Attributen) in der Datenbasis gesucht, welche häufig auftauchen. Dies wird durch den A-priori Algorithmus (oder Abwandlungen dadurch) erreicht. Zu Beginn wird eine mindestens Prozentzahl definiert, wie oft ein Attributmenge auftauchen muss, der sogenannte Minimal-Support. Anschliessend werden für alle 1-elementigen Mengen überpüft, ob sie den . Danach werden diese um ein Attribut erweitert und es wird erneut gezählt.  Wird dieser Wert nicht erreicht, wird die Menge als uninteressant eingestuft und nicht weiter verfolgt. 