
\chapter{Konzept}
\label{sec:konzept}

\section{Vorgehensweise}
\label{sec:konzept:vorgehensweise}
Für die Umsetzung in diesem Projekt eigen Sich die \nameref{sec:recherche:dataminingtechniken:disziplinen:association} (\cref{sec:recherche:dataminingtechniken:disziplinen:association}) sowie die \nameref{sec:recherche:dataminingtechniken:disziplinen:clusteranalysis} (\cref{sec:recherche:dataminingtechniken:disziplinen:clusteranalysis}), welche in diesem Abschnitt weiter beschrieben werden.

\nameref{sec:recherche:dataminingtechniken:disziplinen:classification} (\cref{sec:recherche:dataminingtechniken:disziplinen:classification}) fällt weg, da damit bekannte Klassen zugeteilt werden welche in diesem Projekt jedoch nicht bekannt sind. Die \nameref{sec:recherche:dataminingtechniken:disziplinen:regression} (\cref{sec:recherche:dataminingtechniken:disziplinen:regression}) eignet sich nicht, da dadurch nur numerische Werte vorausgesagt werden können, in dieser Arbeit jedoch Klassen vergeben werden müssen. \nameref{sec:recherche:dataminingtechniken:disziplinen:collaborativefiltering} (\cref{sec:recherche:dataminingtechniken:disziplinen:collaborativefiltering}) versucht durch Kundenbewertungen ähnliche Objekte vorzuschlagen. Dies kann für einen Recommender eingesetzt werden, jedoch nicht für die Auffindung von häufigen Attributen.

%Die Umsetzung in diesem Projekt wird zweistufig durchgeführt. Als erstes gibt der User eine Abfrage ein, für welche anschliessend eine eine Liste von häufig auftretenden Attributen gesucht werden soll (siehe \cref{sec:einletung:ziel} \nameref{sec:einletung:ziel}). Die Abfrage schränkt dabei den Datenbestand ein. Dafür kann der erste Schritt des \nameref{sec:recherche:dataminingtechniken:disziplinen:association} (\cref{sec:recherche:dataminingtechniken:disziplinen:association}) eingesetzt werden. Für die Analyse der Restmenge eignet sich die \nameref{sec:recherche:dataminingtechniken:disziplinen:clusteranalysis} (\cref{sec:recherche:dataminingtechniken:disziplinen:clusteranalysis}), da zu Beginn die Klassen noch nicht bekannt sind. Dadurch fällt \nameref{sec:recherche:dataminingtechniken:disziplinen:classification} (\cref{sec:recherche:dataminingtechniken:disziplinen:classification}) weg. Die \nameref{sec:recherche:dataminingtechniken:disziplinen:regression} (\cref{sec:recherche:dataminingtechniken:disziplinen:regression}) eignet sich nicht, da dadurch nur numerische Werte vorausgesagt werden können, in dieser Arbeit jedoch Klassen vergeben werden müssen. \nameref{sec:recherche:dataminingtechniken:disziplinen:collaborativefiltering} (\cref{sec:recherche:dataminingtechniken:disziplinen:collaborativefiltering}) versucht durch Kundenbewertungen ähnliche Objekte vorzuschlagen. Dies kann für einen Recommender eingesetzt werden, jedoch nicht für die Auffindung von häufigen Attributen.

% Assoziationsanalyse
%Als erstes werden Attributkombinationen (Mengen von Attributen) in der Datenbasis gesucht, welche häufig auftauchen. Dies wird durch den A-priori Algorithmus (oder Abwandlungen dadurch) erreicht. Zu Beginn wird eine mindestens Prozentzahl definiert, wie oft ein Attributmenge auftauchen muss, der sogenannte Minimal-Support. Anschliessend werden für alle 1-elementigen Mengen überpüft, ob sie den . Danach werden diese um ein Attribut erweitert und es wird erneut gezählt.  Wird dieser Wert nicht erreicht, wird die Menge als uninteressant eingestuft und nicht weiter verfolgt. 

\subsection{Einschränkung des Datenbestandes}
Im ersten Schritt wird der Datenbestand durch die Auswahl von Attributen durch den Benutzer eingeschränkt. Dafür eignet sich die Häufigkeitszählung des \nameref{sec:recherche:dataminingtechniken:disziplinen:association} (\cref{sec:recherche:dataminingtechniken:disziplinen:association}), welche in diesem Abschnitt genauer beschrieben wird.

