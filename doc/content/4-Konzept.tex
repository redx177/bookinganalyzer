
\chapter{Konzept}
\label{sec:konzept}

\section{Vorgehensweise}
\label{sec:konzept:vorgehensweise}

% Assoziationsanalyse
Als erstes werden Attributkombinationen (Mengen von Attributen) in der Datenbasis gesucht, welche häufig auftauchen. Dies wird durch den A-priori Algorithmus (oder Abwandlungen dadurch) erreicht. Zu Beginn wird eine mindestens Prozentzahl definiert, wie oft ein Attributmenge auftauchen muss, der sogenannte Minimal-Support. Anschliessend werden für alle 1-elementigen Mengen überpüft, ob sie den . Danach werden diese um ein Attribut erweitert und es wird erneut gezählt.  Wird dieser Wert nicht erreicht, wird die Menge als uninteressant eingestuft und nicht weiter verfolgt. 