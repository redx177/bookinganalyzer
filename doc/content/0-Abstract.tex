% !TeX encoding=utf8
% !TeX spellcheck = de_CH

\subsection*{Kurzfassung}
% Ziel, Grundlagen, Vorgehensweise, Ergebnisse
Das Ziel dieser Arbeit ist es, aus den Buchungsdaten der Interhome AG weiteres Wissen zu gewinnen. Dadurch soll es dem Marketing und dem Einkauf ermöglicht werden, fundiertet Entscheidungen zu treffen. Eine Buchung besteht aus verschiedenen Attributen, wie zum Beispiel die Qualität (Anzahl Sterne), verschiedene Distanzangaben (Distanz bis zum Meer, Skilift, See, etc.), Binärwerte (Haustiere erlaubt?, Grill vorhanden?, Pool vorhanden?, etc.) und vielen mehr. Den Mitarbeitern der Interhome soll ermöglicht werden, dass sie herausfinden, welche Kombinationen solcher Attribute häufig gebucht werden. Als mögliche Antwort kann ein zu entwickelndes Programm zurückliefern, dass von Schweizern oft Objekte in Frankreich gebucht werden, die 500 Meter oder weniger von einem Skilift entfernt sind. Zusätzlich sollen die Daten eingeschränkt werden können, so dass zum Beispiel nur Buchungen von Objekten in der Schweiz analysiert werden. Das Endprodukt soll eine Plattform sein, auf welcher die Interhome selbstständig Analysen auf der Datenbasis durchführen kann, um Wissen über die Buchungen zu generieren.

Um solche Analysen zu ermöglichen, werden in dieser Arbeit verschiedene Techniken des Data Mining vorgestellt und ihre Tauglichkeit für den Einsatz auf den Interhome Daten bewertet. Für die Häufigkeitsanalyse der Attribute wird der Apriori Algorithmus des Association rule learning verwendet. Zusätzlich wird überprüft, ob die Resultate durch den Einsatz von zwei Clustering Algorithmen, k-prototype und DBSCAN, verbessert werden können. Zuerst werden Hypothesen über die Buchungen aufgestellt, die durch die Arbeit entweder bestätigt, oder widerlegt werden sollen. Daraus werden die funktionalen und nicht funktionalen Anforderungen abgeleitet, die von der Umsetzung zu erfüllen sind. Anschliessend wird die Auswahl der oben genannten Algorithmen beschrieben sowie Testfälle für die Validierung der Resultate definiert.  

In der Umsetzung wird eine Weblösung entwickelt, welche von Interhome für die Analyse verwendet werden kann. Dazu werden zuerst die Daten vorbereitet. Stornierte Buchungen werden aussortiert, numerische Werte normiert, zusätzliche Felder hinzugefügt und Diskretierungen durchgeführt. Anschliessend wird der Proof of Concept nach dem aktuellen Stand der Technik entwickelt. Massgeblich beeinflusst wurde die Umsetzung durch die Anforderungen aus dem Konzept. Jedoch auch durch die Laufzeit der eingesetzten Algorithmen, da diese von einigen Sekunden bis zu mehreren Stunden dauern können. Um diese Problematik zu umgehen, wurden verschiedene Datenzugriffe evaluiert, ein Caching eingesetzt und die Berechnung der Algorithmen in Hintergrundprozesse ausgelagert.

Als Resultat werden die Testfälle aus dem Konzept ausgewertet sowie die Hypothesen aus der Einleitung besprochen. Es hat sich gezeigt, dass der Apriori Algorithmus gute Einblicke in die Datenbasis liefern kann. Einige Hypothesen mussten widerlegt werden, wodurch jedoch neue Erkenntnisse der Buchungsdaten gewonnen werden konnten. Der grösste Vorteil dieses Algorithmuses ist seine Geschwindigkeit. Eine Analyse über alle 111'442 Buchungen dauert knapp 10 Sekunden. Dies lädt dazu ein die Daten explorativ zu erforschen. Der Einsatz der Clustering Algorithmen hat sich jedoch nicht bewährt. Es wird vermutet, dass es keine deutlichen Gruppen gibt in den Stammdaten von Interhome gibt. Obwohl dies natürlich einen Rückschlag darstellt, ist es trotzdem eine wichtige Erkenntnis, welche durch diese Arbeit gewonnen werden konnte. 

