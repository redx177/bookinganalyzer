
% Eigener Beitrag: Beschreibung, Begründung, Aufzeigung, Methode, Fazit

\chapter{Anforderungsanalyse}
\label{sec:anforderungsanalyse}

\section{Analyse der Buchungsdaten}

\section{Anforderungen an das Programm}
Im \cref{sec:recherche:programmevaluation} \nameref{sec:recherche:programmevaluation} wurden zwei Programme für das Data Mining vorgestellt. Hier werden die Anforderungen aufgelistet, welche diese Programme zu erfüllen haben, damit sie im weiteren Verlauf dieser Arbeit verwendet werden können.

\subsection{Lizenzen}
Die Applikation sollte kostenlos einsetzbar sein. Sie benötigt demnach eine der folgenden Lizenzmodellen:
\begin{itemize}
\item OpenSource. Software kann verwendet und angepasst werden (siehe \url{https://opensource.org/licenses}).
\item Freeware. Software kann verwendet, jedoch nicht angepasst werden.
\item Studenten Lizenz. Software kann als Student kostenlos eingesetzt werden.
\end{itemize}

\subsection{Betriebsystem}
Die Entwicklung wird privat auf einem Linux Rechner durchgeführt, und Zeitweise auch bei der Arbeit auf Windows Computern welche von Hotelplan Management AG bereitgestellt werden.
Demnach ist es nötig, dass die Software unter Linux sowie auch Windows gestartet werden kann.

\subsection{Funktionsumfang}
Da der Datenbestand 133'001 Einträge umfasst, muss die Plattform und das Lizenzmodel diese Anzahl an Items unterstützen.

Zusätzlich das Programm die im \cref{sec:recherche:dataminingtechniken:disziplinen} \nameref{sec:recherche:dataminingtechniken:disziplinen}) beschriebenen Techniken beherrschen. 

